\documentclass[..docs.tex]{subfiles}

\def\code#1{\texttt{#1}}

\begin{document}

The combinatorics package houses functions related to combinatorics.

\subsection{\code{divisorsInBC}}

\subsubsection{Usage}

The \code{divisorsInBC(n, k, p)} function counts how many factors of $p$ (where $p$ is prime) there are in $\binom{n}{k}.$

\subsubsection{Implementation}

First, some notation. Let $x$ be a positive integer. Define $q_x, r_x$ and $q'_x, r'_x$ (where we have $0 \leq r_x, r'_x < p^x$) as

$$n = p^{x}q_x + r_x,$$
$$k = p^{x}q'_x + r'_x.$$

Note that $q_x, r_x, q'_x, r'_x$ all exist and are unique by the division algorithm.

\vspace{5mm}

Now,  note that $$\binom{n}{k} = \frac{n!}{k!(n-k)!}.$$

To find the number of divisors of $p$ in $\binom{n}{k},$ we will count the divisors of $p$ in the numerator and in the denominator of the above fraction, then subtract.

Note that the desired difference is

$$\left(\lfloor \frac{n}{p} \rfloor + \lfloor \frac{n}{p^2} \rfloor + \cdots \right) - \left(\lfloor \frac{k}{p} \rfloor + \cdots\right) - \left(\lfloor \frac{n-k}{p} + \cdots \rfloor\right).$$

Note that, in the above sums, the $\cdots$ does not indicate (as it normally does) that the sum goes on forever; instead, it only indicates that the sum goes on until one term becomes $0$ (as all subsequent terms will then be $0$). This notation is choosen for succintness.  It's important to distinguish that this is a finite sum, as soon we will do some algebra with it that would be harder to justify with infinite sums.

\vspace{5mm}

Note that

$$\left(\lfloor \frac{n}{p} \rfloor + \lfloor \frac{n}{p^2} \rfloor + \cdots \right) - \left(\lfloor \frac{k}{p} \rfloor + \cdots\right) - \left(\lfloor \frac{n-k}{p} \rfloor + \cdots \rfloor\right) = \left(\lfloor \frac{n}{p} \rfloor - \left(\lfloor \frac{k}{p} \rfloor + \lfloor \frac{n-k}{p} \rfloor \right)\right) + \cdots$$

Thus, it suffices to determine

$$\lfloor \frac{n}{p^x} \rfloor - \left(\lfloor \frac{k}{p^x} \rfloor + \lfloor \frac{n-k}{p^x} \rfloor \right)$$

for each integer $x > 0$ such that $n > p^x.$

\vspace{5mm}

Using the notation defined earlier, note that

$$\lfloor \frac{n}{p^x} \rfloor = q_x,$$
$$\lfloor \frac{k}{p^x} \rfloor = q'_x,$$
$$\lfloor \frac{n-k}{p^x} \rfloor = q_x - q'_x + \lfloor \frac{r_x-r'_x}{p^x} \rfloor.$$

Note that

$$-p^x < r_x - r'_x < p^x,$$

and thus

$$\lfloor \frac{r_x-r'_x}{p^x} \rfloor = 0\text{ or } \lfloor \frac{r_x-r'_x}{p^x} \rfloor = -1.$$

Thus, 

$$\lfloor \frac{n}{p^x} \rfloor - \left(\lfloor \frac{k}{p^x} \rfloor + \lfloor \frac{n-k}{p^x} \rfloor \right)$$

will be either $0$ or $1,$ and it will be $1$ if and only if $r_x < r'_x.$

\vspace{5mm}

Thus, the number of factors of $p$ in $\binom{n}{k}$ is precisely the number of $x$ such that $r_x < r'_x.$ The function simply iterates over each $x$ such that $p^x < n,$ counting whenever we have $r_x < r'_x.$

\subsubsection{Possible Updates}

If I added a prime factorization function to this library, I could use that to extend this method to $p$ that aren't prime.

\vspace{5mm}

It may be more efficient to first compute the largest $x$ such that $p^x < n,$ iterate backwards, then stop as soon as we reach some $y$ such that $r_y < r'_y,$ as all subsequent residues will follow in that pattern (this can be proven by looking at the base $p$ representations of numbers involved). 

\end{document}
